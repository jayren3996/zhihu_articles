\documentclass[10pt,UTF8]{ctexart}
%\CTEXsetup[format={\Large\bfseries}]{section}
\usepackage[T1]{fontenc}
\usepackage[a4paper]{geometry}
\geometry{verbose,tmargin=3cm,bmargin=3cm,lmargin=2cm,rmargin=2cm,headheight=1cm,headsep=1cm}
\usepackage{bm}
\usepackage{amssymb}
\usepackage{amsmath}
\usepackage{setspace}

\makeatletter

\providecommand{\tabularnewline}{\\}
\DeclareRobustCommand\nobreakspace{\leavevmode\nobreak\ }

\makeatother

\begin{document}
\title{Jordan-Wigner变换}
\author{任杰}
\date{}

\maketitle

\section*{自旋与费米子算符的联系}
\subsection*{单粒子情形}
\noindent 单粒子费米子系统是一个简单两能级系统,只有占据与非占据两个状态。分别用 $\left|\uparrow\right\rangle ,\left|\downarrow\right\rangle$ 表示。对于费米子体系,产生湮灭算符为:
\begin{equation}
	\begin{cases}
		c\left|1\right\rangle =\left|0\right\rangle \\ 
		c\left|0\right\rangle =0
	\end{cases},
	\begin{cases}
		c^{\dagger}\left|1\right\rangle =0 \\ 
		c^{\dagger}\left|0\right\rangle =\left|1\right\rangle
	\end{cases}.
\end{equation}
这样,算符的矩阵表示可以写作:
\begin{eqnarray}
	c &=& \left(
	\begin{array}{cc} 
		0 & 0\\ 
		1 & 0 
	\end{array}
	\right)=\sigma^{-}, \\
	c^{\dagger} &=& \left(
	\begin{array}{cc} 
		0 & 1\\ 
		0 & 0 
	\end{array}\right)=\sigma^{+}.
\end{eqnarray}
可以看出费米子产生湮灭算符和自旋系统中的泡利升降算符相同。在单粒子希尔伯特空间中,费米子和泡利算符可以互相转化。
\subsection*{多粒子情形}
\noindent 一般的,量子多体算符/矢量可以看作一些列单体算符/矢量的张量积。如对于一维自旋链,我们可以用单体基矢 $\left|\uparrow\right\rangle ,\left|\downarrow\right\rangle$ 张成整个多体系统的基矢 $\left|\sigma_{1},\sigma_{2},\cdots,\sigma_{N}\right\rangle$ :
\begin{equation}
	\left|\psi\right\rangle =\sum_{\left\{ \sigma_{i}\right\} }\psi\left[\sigma_{1},\sigma_{2},\cdots,\sigma_{N}\right]\left|\sigma_{1},\sigma_{2},\cdots,\sigma_{N}\right\rangle,
\end{equation}
\begin{equation}
	\left|\sigma_{1},\sigma_{2},\cdots,\sigma_{N}\right\rangle =\otimes_{i=1}^{N}\left|\sigma_{i}^{z}\right\rangle 
\end{equation}
类似的,多体算符也可以用单体算符张量积表示,如
\begin{equation}
	\sigma_{i}^{+}=\left(\otimes_{j=1}^{i-1}\mathbb{I}_{j}\right)\otimes\sigma^{+}\otimes\left(\otimes_{j=i+1}^{N}\mathbb{I}_{j}\right)
\end{equation}
这样,对于自旋系统,我们可以用矢量和矩阵直积的方法,将多体问题化为一个 $2^N$ 维线性代数问题。
现在回头看费米子系统,类比自旋链,我们想直接通过直积得到费米子的多体产生湮灭算符表示:
\begin{eqnarray}
	c_{i}&=&\left(\otimes_{j=1}^{i-1}\mathbb{I}_{j}\right)\otimes c\otimes\left(\otimes_{j=i+1}^{N}\mathbb{I}_{j}\right), \\
	c_{i}^{\dagger}&=&\left(\otimes_{j=1}^{i-1}\mathbb{I}_{j}\right)\otimes c^{\dagger}\otimes\left(\otimes_{j=i+1}^{N}\mathbb{I}_{j}\right) 
\end{eqnarray}
需要注意的是,多粒子体系中,由于对易关系的要求,这样的直积表示是错误的。费米子要求的对易关系为:
\begin{equation}
	\left\{ c_{i},c_{j}^{\dagger}\right\} =\delta_{ij},\ \left\{ c_{i},c_{j}\right\} =0.
\end{equation}
可以验证,上述直积表示不满足 $i \ne j$ 时的对易关系。
事实上 $i \ne j$ 时的反对易关系蕴含着费米子算符是一个高度非局域算符,而算符的直积表示只适用于局域的算符。举例来说,对一个两费米子体系:
\begin{equation}
	c_{2}^{\dagger}\left|0,0\right\rangle =\left|0,1\right\rangle,
\end{equation}
\begin{equation}
	c_{2}^{\dagger}\left|1,0\right\rangle =c_{2}^{\dagger}c_{1}^{\dagger}\left|0,0\right\rangle =-c_{1}^{\dagger}c_{2}^{\dagger}\left|0,0\right\rangle =-\left|1,1\right\rangle.
\end{equation}
在1处的费米子占据状态会改变作用于2处的费米子算符的结果。为了得到正确的对易关系,我们引入一非局域链算符
\begin{equation}
	K_{i}=\otimes_{j=1}^{i-1}\left(-\sigma_{j}^{z}\right)=\otimes_{j=1}^{i-1}\left(1-2c_{j}^{\dagger}c_{j}\right) 
\end{equation}
从而将费米子算符写为:
\begin{eqnarray}
	c_{i}&=&K_{i}\otimes c\otimes\left(\otimes_{j=i+1}^{N}\mathbb{I}_{j}\right), \\
	c_{i}^{\dagger}&=&K_{i}\otimes c^{\dagger}\otimes\left(\otimes_{j=i+1}^{N}\mathbb{I}_{j}\right).
\end{eqnarray}
相当与在原来的基础上引入了一条从1 到$i$的链,这条链保证了费米子多体算符的对易性。这也自然引出了自旋算符与费米子算符的变换关系。

\subsection*{Jordan-Wigner 变换}
\noindent Jordan-Wigner 变换是自旋算符与费米子算符之间的转换,可以将一些自旋问题转化为费米子问题。从之前对费米子的讨论中,我们实际上已经得到了 Jordan-Wigner 变换公式:
\begin{equation}
	\begin{cases} \sigma_{i}^{+} & =K_{i}c_{i}^{\dagger}\\ \sigma_{i}^{-} & =K_{i}c_{i} \end{cases},
\end{equation}
\begin{equation}
	\begin{cases} \sigma_{i}^{x} & =K_{i}\left(c_{i}+c_{i}^{\dagger}\right)\\ \sigma_{i}^{y} & =iK_{i}\left(c_{i}-c_{i}^{\dagger}\right)\\ \sigma_{i}^{z} & =2c_{i}^{\dagger}c_{i}-1 \end{cases}
\end{equation}
\begin{equation}
	K_{i}=\prod_{j<i}\left(-\sigma_{j}^{z}\right)=\prod_{j<i}\left(1-2c_{j}^{\dagger}c_{j}\right)
\end{equation}

\section*{例子}
\noindent 下面我们用 Jordan-Wigner 变换处理一个经典自旋模型——横场伊辛模型。该模型哈密顿量为:
\begin{equation}
	\hat{H}=\sum_{i}\sigma_{i}^{x}\sigma_{i+1}^{x}+h\sigma_{i}^{z} .
\end{equation}
经过 Jordan-Wigner 变换后:
\begin{eqnarray}
	\sigma_{i}^{x}\sigma_{i+1}^{x} &=& -\left(c_{i}-c_{i}^{\dagger}\right)\left(c_{i+1}+c_{i+1}^{\dagger}\right), \\
	\sigma_{i}^{z} &=& 2c_{i}^{\dagger}c_{i}-1, \\
	\hat{H} &=& \sum_{i}\left(c_{i}^{\dagger}c_{i+1}+c_{i}^{\dagger}c_{i+1}^{\dagger}+h.c.\right)+h\sum_{i}\left(2c_{i}^{\dagger}c_{i}-1\right).
\end{eqnarray}
由于平移对称性,做傅里叶变换:
\begin{equation}
	\begin{cases} c_{l} & =\frac{1}{\sqrt{n}}\sum_{k}c_{k}e^{+ikl}\\ c_{l}^{\dagger} & =\frac{1}{\sqrt{n}}\sum_{k}c_{k}^{\dagger}e^{-ikl} \end{cases}.
\end{equation}
\begin{equation}
	\hat{H}=\sum_{k}\left(\begin{array}{cc} c_{k}^{\dagger} & c_{-k}\end{array}\right)\left(\begin{array}{cc} h+\cos\left(k\right) & -i\sin\left(k\right)\\ i\sin\left(k\right) & -h-\cos\left(k\right) \end{array}\right)\left(\begin{array}{c} c_{k}\\ c_{-k}^{\dagger} \end{array}\right) 
\end{equation}
再对每个小矩阵对角化,就得到了体系的能谱:
\begin{equation}
	E\left(k\right)=\sqrt{\left(h+\cos k\right)^{2}+\sin^{2}k} 
\end{equation}


\subsection*{参考文献}
\noindent [1] Nagaosa, Quantum field theory in strongly correlated electronic systems. \\
\noindent [2] Xiao-Gang Wen,  Quantum field theory of many-body systems.

\end{document}
