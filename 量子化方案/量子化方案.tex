%% LyX 2.3.4.2 created this file.  For more info, see http://www.lyx.org/.
%% Do not edit unless you really know what you are doing.
\documentclass[UTF8]{ctexart}
\usepackage[T1]{fontenc}
\usepackage{geometry}
\geometry{verbose,tmargin=3cm,bmargin=3cm,lmargin=2cm,rmargin=2cm,headheight=1cm,headsep=1cm}
\usepackage{fancyhdr}
\pagestyle{fancy}
\usepackage{bm}
\usepackage{esint}

\makeatletter
%%%%%%%%%%%%%%%%%%%%%%%%%%%%%% User specified LaTeX commands.
% 如果没有这一句命令,XeTeX会出错,原因参见
% http://bbs.ctex.org/viewthread.php?tid=60547
\DeclareRobustCommand\nobreakspace{\leavevmode\nobreak\ }

\makeatother

\begin{document}
\title{多体系统的量子化方案}
\date{任杰}

\maketitle
对于多体系统,有两种平行的量子化方案:正则量子化和路径积分量子化。相比而言,路径积分有更简洁优雅的形式,而正则量子化有着“更严格”的量子力学结构。本文希望做到两点:
\begin{itemize}
\item 更详细地讨论两种框架下定义中一些容易忽略的小细节。
\item 将两个框架的计算放在一起,比较它们的相同与不同之处。
\end{itemize}

\section{傅立叶变换}

两种框架内处理的模型基本都在实空间定义,但在计算时变换到动量和频率空间。为了不弄乱变换前后的系数,这里总结对场算符和格林函数的傅立叶变换(它们的系数是不同的!)。大多数多体物理书中都采用以下的系数。 

\subsection{场算符变换}

首先在薛定谔绘景下,场算符不含时,其傅立叶变换采用“开根号”归一化:

\begin{equation}
\hat{\psi}\left(x\right)=\frac{1}{\sqrt{V}}\sum_{\bm{k}}e^{+i\bm{k}\cdot\bm{x}}\hat{c}_{k},
\end{equation}

\begin{equation}
\hat{c}_{k}=\frac{1}{\sqrt{V}}\int d^{d}xe^{-i\bm{k}\cdot\bm{x}}\hat{\psi}\left(\bm{x}\right).
\end{equation}
其中算符 $c_{\bm{k}}$ 是正则量子化中的 $\bm{k}$ 动量粒子湮灭算符,满足正则对易关系

\begin{equation}
\left[\hat{c}_{\bm{k}},\hat{c}_{\bm{k}'}^{\dagger}\right]_{\zeta}=\delta_{\bm{k}\bm{k}'}.
\end{equation}
该算符是严格的量子力学算符,场算符 $\psi\left(\bm{x}\right)$ 可看出此湮灭算符的傅立叶变换。

引入时间(虚时间)演化后,海森堡绘景下场算符定义为

\begin{equation}
\hat{c}_{\bm{k}}\left(\tau\right)=e^{+\hat{H}\tau}\hat{c}_{\bm{k}}e^{-\hat{H}\tau},
\end{equation}

\begin{equation}
\hat{\psi}\left(\bm{x},\tau\right)=e^{+\hat{H}\tau}\hat{\psi}\left(\bm{x}\right)e^{-\hat{H}\tau}.
\end{equation}
正则量子化下,算符往往只变换到动量空间而非频率空间。而路径积分方案中情况不同,我们会通过相干态表象将场算符变为一个数

\begin{equation}
\hat{\psi}\left(\bm{x},\tau\right)\left|\psi\right\rangle =\psi\left(\bm{x},\tau\right)\left|\psi\right\rangle .
\end{equation}
计算路径积分时我们往往将此数变换到动量、频率空间:

\begin{equation}
\psi\left(\bm{x},\tau\right)=\frac{1}{\sqrt{\beta V}}\sum_{\bm{k}}\sum_{n}e^{+ik\cdot x}\psi_{\bm{k},n},
\end{equation}

\begin{equation}
\psi_{\bm{k},n}=\frac{1}{\sqrt{\beta V}}\int_{0}^{\beta}d\tau\int d^{d}x\ e^{-ik\cdot x}\psi\left(\bm{x},\tau\right).
\end{equation}
变换归一化系数和算符定义是一样的,只是这里的量均是数而非算符。

\subsection{格林函数变换}

与场算符不同,函数(包括格林函数、势函数等)的傅立叶变换为的归一化为:

\begin{equation}
f\left(\bm{x}\right)=\frac{1}{V}\sum_{\bm{k}}e^{+i\bm{k}\cdot\bm{x}}f_{\bm{k}},
\end{equation}

\begin{equation}
f_{\bm{k}}=\int d^{d}xe^{-i\bm{k}\cdot\bm{x}}f\left(\bm{x}\right).
\end{equation}
这样归一化下,热力学极限(连续极限)的傅立叶变换变为:

\begin{equation}
f\left(\bm{x}\right)=\int\frac{d^{d}k}{\left(2\pi\right)^{d}}e^{+i\bm{k}\cdot\bm{x}}f_{\bm{k}}.
\end{equation}
引入虚时间变量,同时总是假设哈密顿量不含时间,平移不变。此时格林函数为:

\begin{equation}
G\left(\bm{x},\tau;\bm{x}',\tau'\right)=G\left(\bm{x}-\bm{x}';\tau-\tau'\right).
\end{equation}
其傅立叶变换为:

\begin{equation}
G\left(\bm{x},\tau;\bm{x}',\tau'\right)=\frac{1}{\beta V}\sum_{\bm{k}}\sum_{n}e^{+ik\cdot x}G_{\bm{k},n},
\end{equation}

\begin{equation}
G_{\bm{k},n}=\int d^{d}x\int_{0}^{\beta}d\tau\ e^{-ik\cdot x}G\left(\bm{x};\tau\right).
\end{equation}


\section{自由场格林函数}

\subsection{格林函数定义}

无论何种量子化方案下,格林函数定义都可写为:

\begin{equation}
G\left(\bm{x},\tau;\bm{x}',\tau'\right)=-\left\langle \hat{\psi}\left(\bm{x},\tau\right)\hat{\psi}^{\dagger}\left(\bm{x}',\tau'\right)\right\rangle _{\tau},
\end{equation}
其中 $\left\langle \cdots\right\rangle _{\tau}$ 代表取时序的热力学平均值。这个平均在不同量子化方案下有区别。我们所有的计算都会在动量空间完成,因此目标是得到动量、频率空间的格林函数
$G_{\bm{k},n}$.

\subsubsection{正则量子化}

正则量子化方案下:
\begin{equation}
\left\langle \cdots\right\rangle _{\tau}:=-\frac{1}{\mathcal{Z}}Tr\left[e^{-\beta H}T_{\tau}\left(\cdots\right)\right],
\end{equation}
其中配分函数为:

\begin{equation}
\mathcal{Z}=Tr\left[e^{-\beta H}\right].
\end{equation}
空间傅立叶变换在两种方案下是一致的,格林函数定义包含场算符二次型,因此格林函数的傅立叶系数分解成两个“开平方系数”恰好和场算符傅立叶系数相容:

\begin{equation}
G_{\bm{k}}\left(\tau-\tau'\right)=-\left\langle \hat{c}_{\bm{k}}\left(\tau\right)\hat{c}_{\bm{k}}^{\dagger}\left(\tau'\right)\right\rangle _{\tau},
\end{equation}
频率空间格林函数表达为:

\begin{equation}
G_{\bm{k},n}=\int_{0}^{\beta}d\tau e^{i\omega_{n}\tau}G_{\bm{k}}\left(\tau\right).
\end{equation}


\subsubsection{路径积分量子化}

路径积分量子化下:

\begin{equation}
\left\langle \cdots\right\rangle _{\tau}:=-\frac{1}{\mathcal{Z}}\int\mathcal{D}\left[\psi^{\dagger},\psi\right]e^{-S\left[\psi^{\dagger},\psi\right]}\left(\cdots\right),
\end{equation}
其中作用量为
\begin{equation}
S\left[\psi^{\dagger},\psi\right]=\int d^{d}x\int_{0}^{\beta}\tau\left[\psi^{\dagger}\partial_{\tau}\psi+\hat{H}-\mu\hat{N}\right],
\end{equation}
配分函数为:

\begin{equation}
\mathcal{Z}=\int\mathcal{D}\left[\psi^{\dagger},\psi\right]e^{-S\left[\psi^{\dagger},\psi\right]}.
\end{equation}
注意路径积分量子化下,我们会将算符 $\hat{\psi}$ 换为其在相干态下的本征值 $\psi$, 这是一个交换或反交换(Grassmann)数。由于场泛函积分自动是编时的,频率格林函数可以简单写为:

\begin{equation}
G_{\bm{k},n}=-\left\langle \psi_{\bm{k},n}\bar{\psi}_{\bm{k},n}\right\rangle _{\tau}=-\zeta\left\langle \bar{\psi}_{\bm{k},n}\psi_{\bm{k},n}\right\rangle _{\tau}.
\end{equation}


\subsection{自由场格林函数计算}

首先假定自由场哈密顿量已经对角化为:

\begin{equation}
\hat{H}=\sum_{\bm{k}}\sum_{\alpha}\epsilon_{\bm{k}\alpha}\hat{c}_{\bm{k}\alpha}^{\dagger}\hat{c}_{\bm{k}\alpha}.
\end{equation}
此时海森堡绘景下粒子算符的时间演化为:

\begin{eqnarray}
\hat{c}_{\bm{k}\alpha}\left(\tau\right) & = & e^{-\left(\epsilon_{\bm{k}\alpha}-\mu\right)\tau}\hat{c}_{\bm{k}\alpha},\\
\hat{c}_{\bm{k}\alpha}^{\dagger}\left(\tau\right) & = & e^{+\left(\epsilon_{\bm{k}\alpha}-\mu\right)\tau}\hat{c}_{\bm{k}\alpha}^{\dagger}.
\end{eqnarray}
而相干态表象下,哈密顿量在场构形 $\psi$ 下的值为:

\begin{equation}
H\left[\bar{\psi},\psi\right]=\sum_{\bm{k}}\sum_{\alpha}\epsilon_{\bm{k}\alpha}\bar{\psi}_{\bm{k}\alpha}\psi_{\bm{k}\alpha}.
\end{equation}
拉氏量为:

\begin{equation}
S\left[\bar{\psi},\psi\right]=\sum_{\bm{k}}\sum_{\alpha}\sum_{n}\bar{\psi}_{\bm{k}\alpha,n}\left[-i\omega_{n}+\epsilon_{\bm{k}\alpha}-\mu\right]\psi_{\bm{k}\alpha,n}.
\end{equation}


\subsubsection{正则量子化}

对自由场,格林函数:
\begin{equation}
G_{\bm{k}\alpha}\left(\tau\right)=-\theta\left(\tau\right)e^{-\left(\epsilon_{\bm{k}\alpha}-\mu\right)\tau}n_{\zeta}\left(\epsilon_{\bm{k}\alpha}\right)-\theta\left(-\tau\right)e^{+\left(\epsilon_{\bm{k}\alpha}-\mu\right)\tau}\left(1-n_{\zeta}\left(\epsilon_{\bm{k}\alpha}\right)\right).
\end{equation}
傅立叶变换得到:

\begin{eqnarray}
G_{\bm{k}\alpha,n} & = & \int_{0}^{\beta}d\tau\ e^{i\omega_{n}\tau}G_{\bm{k}\alpha}\left(\tau\right)\nonumber \\
 & = & -\frac{1}{e^{\beta\left(\epsilon_{\bm{k}\alpha}-\mu\right)}-\zeta}\int_{0}^{\beta}d\tau\ e^{\left(i\omega_{n}-\epsilon_{\bm{k}\alpha}+\mu\right)\tau}\nonumber \\
 & = & \frac{1}{i\omega_{n}-\epsilon_{\bm{k}\alpha}+\mu}.
\end{eqnarray}


\subsubsection{路径积分量子化}

先计算生成函数:

\begin{equation}
\mathcal{Z}_{\alpha}\left[\bar{J}_{\bm{k},n},J_{\bm{k},n}\right]=\int d\left(\bar{\psi}_{\bm{k}\alpha,n},\psi_{\bm{k}\alpha,n}\right)\exp\left[-\bar{\psi}_{\bm{k}\alpha,n}\left(-i\omega+\epsilon_{\bm{k}\alpha}-\mu\right)\psi_{\bm{k}\alpha,n}-\bar{\psi}_{\bm{k}\alpha,n}J_{\bm{k},n}-\bar{J}_{\bm{k},n}\psi_{\bm{k}\alpha,n}\right].
\end{equation}
积分测度由高斯积分

\begin{equation}
\int d\left(\bar{\psi}_{n},\psi_{n}\right)e^{-\bar{\psi}_{n}\epsilon\psi_{n}}=\left(\beta\epsilon\right)^{-\zeta}.
\end{equation}
由此得到生成函数为

\begin{equation}
\mathcal{Z}_{\alpha}\left[\bar{J}_{\bm{k},n},J_{\bm{k},n}\right]=\left[\beta\left(-i\omega+\epsilon_{\bm{k}\alpha}-\mu\right)\right]^{-\zeta}\exp\left(-\frac{\bar{J}_{\bm{k},n}J_{\bm{k},n}}{i\omega-\epsilon_{\bm{k}\alpha}+\mu}\right).
\end{equation}
格林函数为

\begin{eqnarray}
G_{\bm{k}\alpha,n} & = & -\frac{1}{\mathcal{Z}\left[0\right]}\left.\frac{\partial^{2}\mathcal{Z}\left[\bar{J}_{\bm{k},n},J_{\bm{k},n}\right]}{\partial\bar{J}_{\bm{k},n}\partial J_{\bm{k},n}}\right|_{\bar{J}_{\bm{k},n},J_{\bm{k},n}=0}\nonumber \\
 & = & \frac{1}{i\omega-\epsilon_{\bm{k}\alpha}+\mu}.
\end{eqnarray}

\end{document}
