\documentclass[UTF8]{ctexart}
\usepackage[T1]{fontenc}
\usepackage{geometry}
\geometry{verbose,tmargin=3cm,bmargin=4cm,lmargin=2cm,rmargin=2cm,headheight=1cm,headsep=1cm}
\usepackage{amsmath}
\usepackage{amssymb}

\makeatletter
\DeclareRobustCommand\nobreakspace{\leavevmode\nobreak\ }

\makeatother

\begin{document}
\title{群同调与上同调简介}
\author{任杰}
\date{}

\maketitle
\noindent
本文将给出群同调与上同调的基本定义,并计算几个最简单的例子,希望给群同调与上同调一个直观的认识。

\section*{Free resolution}
\noindent
我们先给出形式再做说明。Free resolution 是一个正合序列(exact sequence):
\begin{equation}
\begin{array}{cccccccccc}
\stackrel{\cdots}{\longrightarrow} & F_{2} & \stackrel{\partial_{2}}{\longrightarrow} & F_{1} & \stackrel{\partial_{1}}{\longrightarrow} & F_{0} & \stackrel{\partial_{0}}{\longrightarrow} & \mathcal{M} & \stackrel{0}{\longrightarrow} & 0\end{array}
\end{equation}
其中 $\mathcal{M}$ 为 G-module,定义为一个阿贝尔群,并且有群作用:
\begin{equation}
m\in\mathcal{M}\Rightarrow g\cdot m\in\mathcal{M},\forall g\in G.
\end{equation}
正合序列意味着
\begin{equation}
\ker\left(\partial_{i}\right)=imag\left(\partial_{i+1}\right).
\end{equation}
设 $\mathcal{M}$ 由一组基底 $\left\{ m_{1},m_{2},\cdots,m_{k}\right\} $
生成(但不一定是自由生成),要使得序列正合,可定义一个 R-module $F_{1}$ :
\begin{equation}
F_{0}=\mathcal{R}^{k}=\mathcal{R}e_{1}\oplus\cdots\oplus\mathcal{R}e_{k}
\end{equation}
其中 $\mathcal{R}$ 为一个 group-ring:
\begin{equation}
\mathcal{R}=\mathbb{Z}\left[G\right]=\sum_{g\in G}c_{g}\cdot g,\ c_{g}\in\mathbb{Z}.
\end{equation}
此时 $F_{0}$ 由 $\left\{ e_{1},\cdots,e_{k}\right\} $ 自由生成,其中同态映射 $\partial_{0}$
效果为:
\begin{eqnarray}
e_{i} & \mapsto & m_{i}\\
re_{i} & \mapsto & r\cdot m_{i}
\end{eqnarray}
设该映射的核 $\ker\left(\partial_{0}\right)$ 一组基为 $\left\{ b_{1},\cdots,b_{l}\right\} $,
同样 $\ker\left(\partial_{0}\right)$ 不一定由这组基自由生成,但同样可以构造
\begin{equation}
F_{1}=\mathcal{R}^{l}=\mathcal{R}c_{1}\oplus\cdots\oplus\mathcal{R}c_{l}
\end{equation}
而 $\partial_{1}:\ c_{i}\mapsto b_{i}$, 以此类推。Free resolution 就是这样一个将
G-module 写开成一个自由群正合序列的过程。最终我们只保留阶数不小于0的部分:
\begin{equation}
\begin{array}{cccccccc}
F_{*}: & \cdots & \stackrel{\partial_{3}}{\longrightarrow} & F_{2} & \stackrel{\partial_{2}}{\longrightarrow} & F_{1} & \stackrel{\partial_{1}}{\longrightarrow} & F_{0}\end{array}
\end{equation}
注意 Free resolution 的选取不唯一,但可以证明不同的选取只相差一个链同伦,不会影响后面的结果。
\subsection*{例: bar resolution}
\noindent
Bar resolution 是一个一般的 free resolution 构造方法。考虑 $\mathcal{M}=\mathbb{Z}$,
G 在 module $\mathcal{M}$ 上作用平凡:$g\cdot1=1$, 
\begin{equation}
\begin{array}{cccccccccc}
\stackrel{\cdots}{\longrightarrow} & F_{2} & \stackrel{\partial_{2}}{\longrightarrow} & F_{1} & \stackrel{\partial_{1}}{\longrightarrow} & F_{0} & \stackrel{\partial_{0}}{\longrightarrow} & \mathbb{Z} & \stackrel{0}{\longrightarrow} & 0\end{array}
\end{equation}
此时各阶 R-module 有一组简单基底
\begin{eqnarray}
F_{0} & = & \mathcal{R}\left[\ \right]\\
\left[\ \right] & \stackrel{\partial_{0}}{\longmapsto} & 1
\end{eqnarray}
此时 $\ker\partial_{0}=span\left\{ g\left[\ \right]-\left[\ \right],\forall g\in G\right\} $,
因此再定义
\begin{eqnarray}
F_{1} & = & \bigoplus_{g\in G}\mathcal{R}\left[g\right]\\
\left[g\right] & \stackrel{\partial_{1}}{\longmapsto} & g\left[\ \right]-\left[\ \right]
\end{eqnarray}
可以验证此时 $\ker\partial_{1}=span\left\{ g\left[h\right]-\left[gh\right]+\left[g\right],\forall g,h\in G\right\} $,再定义
\begin{eqnarray}
F_{2} & = & \bigoplus_{g,h\in G}\mathcal{R}\left[\left.g\right|h\right]\\
\left[\left.g\right|h\right] & \stackrel{\partial_{2}}{\longmapsto} & g\left[h\right]-\left[gh\right]+\left[g\right]
\end{eqnarray}
这个过程可以一只做下去,这组基底一般形式为
\begin{equation}
\left[g_{1}|g_{2}|\cdots|g_{n}\right]\stackrel{\partial_{n}}{\longmapsto}g_{1}\left[g_{2}|\cdots|g_{n}\right]-\left[g_{1}g_{2}|g_{3}|\cdots|g_{n}\right]+\left[g_{1}|g_{2}g_{3}|\cdots|g_{n}\right]+\cdots+\left[g_{1}|g_{2}|\cdots|g_{n-1}\right]
\end{equation}
注意到
\begin{equation}
e\left[\ \right]-\left[\ \right]=0
\end{equation}
即 $e\left[\ \right]-\left[\ \right]$ 实际上不在 $\ker\partial_{1}$ 中。以此类推,我们可以丢掉
bar resolution 中所有形如 $\left[g_{1}|\cdots|e|\cdots|g_{n}\right]$ 的基底。以上构造是一般的。但这个基底往往有很多冗余,因此具体计算几乎不会使用这组基底。

\section*{群同调}
\noindent
要定义群的同调,还需要一步 co-invariant 操作
\begin{equation}
F_{i}\mapsto F_{i}/\left(m=g\cdot m\right)
\end{equation}
对于 free module, 模掉这个等价关系相当于把系数 $\mathcal{R}$ 换为 $\mathbb{Z}$, 对于上述的
bar resolution, co-invariant 操作后的链序列为:
\begin{equation}
\begin{array}{ccccccccccc}
\mathcal{F}_{*}: & \stackrel{\cdots}{\longrightarrow} & {\displaystyle \bigoplus_{g,h\in G}\mathbb{Z}\left[\left.g\right|h\right]} & \stackrel{\partial_{2}}{\longrightarrow} & {\displaystyle \bigoplus_{g\in G}\mathbb{Z}\left[g\right]} & \stackrel{\partial_{1}}{\longrightarrow} & \mathbb{Z}\left[\ \right] & \stackrel{\partial_{0}}{\longrightarrow} & \mathbb{Z} & \stackrel{0}{\longrightarrow} & 0\end{array}
\end{equation}
群 G 的\textbf{同调群}定义为序列 $\mathcal{F}_{*}$ 的同调群
\begin{equation}
H_{n}\left(G,\mathbb{Z}\right)\equiv H_{n}\left(\mathcal{F}_{*}\right)
\end{equation}


\subsection*{例:$G=\mathbb{Z}_{2}$:}
\noindent
对于这个简单的群,可以直接使用 bar resolution 计算群同调。此时 $G=\left\{ e,\tau\right\} ,\tau^{2}=1,$
各阶 Z module 都只有一个基底 $\left[\tau|\cdots|\tau\right]$:
\begin{equation}
\begin{array}{cccccccccc}
\stackrel{\cdots}{\longrightarrow} & {\displaystyle \mathbb{Z}\left[\left.\tau\right|\tau\right]} & \stackrel{\partial_{2}}{\longrightarrow} & {\displaystyle \mathbb{Z}\left[\tau\right]} & \stackrel{\partial_{1}}{\longrightarrow} & \mathbb{Z}\left[\ \right] & \stackrel{\partial_{0}}{\longrightarrow} & \mathbb{Z} & \stackrel{0}{\longrightarrow} & 0\end{array}
\end{equation}
各阶映射对应基变换为:
\begin{eqnarray}
\left[\ \right] & \mapsto & 1\\
\left[\tau\right] & \mapsto & \left[\ \right]-\left[\ \right]=0\\
\left[\left.\tau\right|\tau\right] & \mapsto & \left[\tau\right]-\left[e\right]+\left[\tau\right]=2\left[\tau\right]
\end{eqnarray}
其一般形式为:
\begin{equation}
\left[\tau|\cdots|\tau\right]_{n}\mapsto\begin{cases}
1 & n=0\\
0 & n=1\mod2\\
2\left[\tau|\cdots|\tau\right]_{n-1} & n=0\mod2
\end{cases}
\end{equation}
因此经过 co-invariant 操作后的序列为:
\begin{equation}
\begin{array}{cccccccccc}
\mathcal{F}_{*}: & \cdots & \stackrel{2}{\longrightarrow} & \mathbb{Z} & \stackrel{0}{\longrightarrow} & {\displaystyle \mathbb{Z}} & \stackrel{2}{\longrightarrow} & {\displaystyle \mathbb{Z}} & \stackrel{0}{\longrightarrow} & \mathbb{Z}\end{array}
\end{equation}
读出 $\mathbb{Z}_{2}$ 群的同调群为
\begin{equation}
H_{n}\left(\mathbb{Z}_{2},\mathbb{Z}\right)=\begin{cases}
\mathbb{Z} & n=0\\
\mathbb{Z}_{2} & n=1\mod2\\
0 & n=0\mod2
\end{cases}
\end{equation}


\section*{群上同调}
\noindent
群上同调定义是类似的,通过对偶作用:
\begin{equation}
\begin{array}{ccccccc}
\mathcal{F}_{*}: & \stackrel{\cdots}{\longrightarrow} & F_{2} & \stackrel{\partial_{2}}{\longrightarrow} & F_{1} & \stackrel{\partial_{1}}{\longrightarrow} & F_{0}\\
\downarrow &  & \downarrow &  & \downarrow &  & \downarrow\\
\mathcal{F}^{*}: & \stackrel{\cdots}{\longleftarrow} & \hom\left(F_{2},\mathcal{M}\right) & \stackrel{d_{1}}{\longleftarrow} & \hom\left(F_{1},\mathcal{M}\right) & \stackrel{d_{0}}{\longleftarrow} & \hom\left(F_{0},\mathcal{M}\right)
\end{array}
\end{equation}
群上同调定义为:
\begin{equation}
H^{n}\left(G,\mathcal{M}\right)\equiv H^{n}\left(\mathcal{F}^{*}\right)
\end{equation}
注意群上同调依赖 G-module 的选取,$\mathcal{M}$ 也被称做群上同调的系数(group cohomology
with coefficient)。对于自由群而言,其对偶空间为:
\begin{equation}
\hom\left(\mathbb{Z}^{k},\mathcal{M}\right)=\mathcal{M}^{k}.
\end{equation}
而上边缘算子 $d_{i}$ 和下边缘算子 $\partial_{i+1}$ 是分别对应的:
\begin{equation}
\left\langle d_{i}C^{i},C_{i+1}\right\rangle =\left\langle C^{i},\partial_{i+1}C_{i+1}\right\rangle 
\end{equation}


\subsection*{例1:$H^{n}\left(\mathbb{Z}_{k},\mathbb{Z}\right)$ :}
\noindent
这里 $\mathbb{Z}$ 是群 $\mathbb{Z}_{k}$ 平凡作用的 module. 对循环群,由于只有一个生成元,此时计算可以大大简化。循环群的
group-ring 为:
\begin{equation}
\mathcal{R}=\mathbb{Z}\left[G\right]=\mathbb{Z}\left[t\right]/\left(t^{k}-1\right)
\end{equation}
其中等价关系 $t^{k}-1=0$ 可分解为:
\begin{equation}
t^{k}-1=\left(t-1\right)\left(t^{k-1}+t^{k-2}+\cdots+1\right)=0
\end{equation}
构造 $F_{0}$ 与 $\partial_{0}$ :
\begin{eqnarray}
\mathcal{R}e_{0} & \rightarrow & \mathbb{Z}\\
e_{0} & \mapsto & 1
\end{eqnarray}
注意到此时
\begin{equation}
\ker\partial_{0}=span\left\{ \left(t-1\right)e_{0}\right\} 
\end{equation}
根据等价关系的分解,我们可以继续构造只包含一个基底的 $F_{1}$ 与 $\partial_{1}$ :
\begin{eqnarray}
\mathcal{R}e_{1} & \rightarrow & \mathcal{R}e_{1}\\
e_{1} & \mapsto & \left(t-1\right)e_{0}
\end{eqnarray}
此时
\begin{equation}
\ker\partial_{1}=span\left\{ Ne_{1}\right\} ,\ N=t^{k-1}+t^{k-2}+\cdots+1
\end{equation}
更高阶的 $F_{i}$ 可按此方法周期构造,最终经过 co-invariant 操作后的序列为:
\begin{equation}
\begin{array}{cccccccccccc}
\mathcal{F}_{*}: & \cdots & \stackrel{t-1}{\longrightarrow} & \mathcal{R} & \stackrel{N}{\longrightarrow} & {\displaystyle \mathcal{R}} & \stackrel{t-1}{\longrightarrow} & \mathcal{R} & \stackrel{1}{\longrightarrow} & \mathbb{Z} & \stackrel{0}{\longrightarrow} & 0\end{array}
\end{equation}
对偶作用后:
\begin{equation}
\begin{array}{ccccccccc}
\mathcal{F}_{*}: & \rightarrow & \mathcal{R} & \stackrel{t-1}{\longrightarrow} & \mathcal{R} & \stackrel{N}{\longrightarrow} & \mathcal{R} & \stackrel{t-1}{\longrightarrow} & \mathcal{R}\\
\downarrow &  & \downarrow &  & \downarrow &  & \downarrow &  & \downarrow\\
\mathcal{F}^{*}: & \leftarrow & \mathbb{Z} & \stackrel{0}{\longleftarrow} & \mathbb{Z} & \stackrel{k}{\longleftarrow} & \mathbb{Z} & \stackrel{0}{\longleftarrow} & \mathbb{Z}
\end{array}
\end{equation}
因为群在 $\mathbb{Z}$ 上作用是平凡的,$t-1\mapsto0,N\mapsto k$,
\begin{equation}
H^{n}\left(\mathbb{Z}_{k},\mathbb{Z}\right)=\begin{cases}
\mathbb{Z} & n=0\\
0 & n=1\mod2\\
\mathbb{Z}_{k} & n=0\mod2
\end{cases}
\end{equation}


\subsection*{例2:$H^{n}\left(\mathbb{Z}_{k},U\left(1\right)\right)$ :}
\noindent
这里 $\mathbb{Z}_{k}$ 在 $U\left(1\right)$ 上作用也是平凡的。我们希望群乘是加法,因此将 $U(1)$
写作 $\mathbb{R}/\mathbb{Z}$:
\begin{equation}
H^{n}\left(\mathbb{Z}_{k},U\left(1\right)\right)\approx H^{n}\left(\mathbb{Z}_{k},\mathbb{R}/\mathbb{Z}\right)
\end{equation}
这样我们只需将例1的结果稍加修改得到:
\begin{equation}
\begin{array}{ccccccccc}
\mathcal{F}_{*}: & \rightarrow & \mathcal{R} & \stackrel{t-1}{\longrightarrow} & \mathcal{R} & \stackrel{N}{\longrightarrow} & \mathcal{R} & \stackrel{t-1}{\longrightarrow} & \mathcal{R}\\
\downarrow &  & \downarrow &  & \downarrow &  & \downarrow &  & \downarrow\\
\mathcal{F}^{*}: & \leftarrow & \mathbb{R}/\mathbb{Z} & \stackrel{0}{\longleftarrow} & \mathbb{R}/\mathbb{Z} & \stackrel{k}{\longleftarrow} & \mathbb{R}/\mathbb{Z} & \stackrel{0}{\longleftarrow} & \mathbb{R}/\mathbb{Z}
\end{array}
\end{equation}
这时和整数情况稍有不同,简单计算得到上同调群为:
\begin{equation}
H^{n}\left(\mathbb{Z}_{k},U\left(1\right)\right)=\begin{cases}
\mathbb{R}/\mathbb{Z}=U\left(1\right) & n=0\\
\mathbb{Z}_{k} & n=1\mod2\\
0 & n=0\mod2
\end{cases}
\end{equation}


\subsection*{例3:$H^{n}\left(\mathbb{Z}_{k}\times\mathbb{Z}_{T},U_{T}\left(1\right)\right)$:}
\noindent
其中 $\mathbb{Z}_{T}=\left\{ e,\tau\right\} $ 是时间反演算符,在 module $U_{T}(1)$
上的作用为:
\begin{eqnarray}
e\cdot1 & = & 1\\
\tau\cdot1 & = & -1
\end{eqnarray}
我们可以将 $\mathbb{Z}_{k}\times\mathbb{Z}_{T}$ 写作 $\mathbb{Z}_{2k}$, 
\begin{equation}
H^{n}\left(\mathbb{Z}_{k}\times\mathbb{Z}_{T},U_{T}\left(1\right)\right)\approx H^{n}\left(\mathbb{Z}_{2k},\mathbb{R}/\mathbb{Z}\right)
\end{equation}
此时群的唯一生成元为 $\tau$, $\tau$ 在 $U_{T}(1)$ 上作用将单位元取逆。因此 $t-1\mapsto-1,N\mapsto0$,
\begin{equation}
\begin{array}{ccccccccc}
\mathcal{F}_{*}: & \rightarrow & \mathcal{R} & \stackrel{t-1}{\longrightarrow} & \mathcal{R} & \stackrel{N}{\longrightarrow} & \mathcal{R} & \stackrel{t-1}{\longrightarrow} & \mathcal{R}\\
\downarrow &  & \downarrow &  & \downarrow &  & \downarrow &  & \downarrow\\
\mathcal{F}^{*}: & \leftarrow & \mathbb{R}/\mathbb{Z} & \stackrel{-2}{\longleftarrow} & \mathbb{R}/\mathbb{Z} & \stackrel{0}{\longleftarrow} & \mathbb{R}/\mathbb{Z} & \stackrel{-2}{\longleftarrow} & \mathbb{R}/\mathbb{Z}
\end{array}
\end{equation}
稍加计算得到:
\begin{equation}
H^{n}\left(\mathbb{Z}_{k}\times Z_{T},U_{T}\left(1\right)\right)=\begin{cases}
0 & n=1\mod2\\
\mathbb{Z}_{2} & n=0\mod2
\end{cases}
\end{equation}

\section*{Remark}
\noindent
以上的例子可以展示出群上同调计算的基本流程。但实际上对于多个生成元的群,很难像循环群那样轻易地构造出简单的 free resolution,这时群上同调计算困难的地方。

\end{document}
