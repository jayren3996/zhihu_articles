\documentclass[UTF8]{ctexart}
\usepackage[T1]{fontenc}
\usepackage{geometry}
\geometry{verbose,tmargin=3cm,bmargin=3cm,lmargin=2cm,rmargin=2cm,headheight=1cm,headsep=1cm}
\usepackage{amsmath}
\usepackage{bm}
\usepackage{esint}

\def\tr{\mathrm{tr}}
\def\Tr{\mathrm{Tr}}

\makeatletter
\DeclareRobustCommand\nobreakspace{\leavevmode\nobreak\ }

\makeatother

\begin{document}
\title{路径积分量子化}
\author{任杰}
\date{}

\maketitle

\section*{路径积分的建立}
\subsection*{拆分配分函数}
\noindent 
考虑 $d$ 维格电系统,格点数为 $N^d$,哈密顿量可表达为格点上升降算符函数:
\begin{equation}
	\hat H = H(\hat c_i^\dagger, \hat c_j).
\end{equation}
场论中的核心问题是计算体系的热力学配分函数,即:
\begin{equation}
	Z \equiv \Tr(e^{-\beta \hat H} ).
\end{equation}
以上求迹表达式中,逆温度 $\beta$ 可拆分为 $M$ 等份,求迹表达式变为:
\begin{equation}
	Z = \Tr \left[e^{-\frac{\beta}{M}\hat H} \cdots e^{-\frac{\beta}{M}\hat H} \right].
\end{equation}
求迹过程可引入任意一组完备基底 $|\psi_\alpha \rangle $,要求:
\begin{equation}
	\sum_{\alpha} |\psi_\alpha \rangle\langle \psi_\alpha | = \bm 1.
\end{equation}
这样,配分函数可变为矩阵连乘形式:
\begin{equation}
	Z = \sum_{\alpha_1,\cdots, \alpha_M} \mathcal M_{\alpha_1,\alpha_2} \cdots \mathcal M_{\alpha_M,\alpha_1}.
\end{equation}
其中 $\mathcal M$ 的矩阵元为:
\begin{equation}
	\mathcal M_{\alpha,\beta} \equiv 
	\exp \left[ - \frac{\beta}{M} \langle\psi_\alpha| \hat H |\psi_\beta \rangle \right]
	\simeq 1- \frac{\beta}{M} \langle\psi_\alpha| \hat H |\psi_\beta \rangle.
\end{equation}
路径积分中采用相干态作为(超)完备基底。这里相干态可以指波色相干态、费米相干态以及自旋相干态。它们的性质略有不同,下面分情况讨论。



\subsection*{波色相干态}
\noindent 波色场相干态(未归一化)定义为:
\begin{equation}
	|\phi\rangle \equiv \exp\left(\sum_i \phi_i \hat c_i^\dagger \right)|0\rangle.
\end{equation}
这里态的标记 $\phi$ 是一个 $N^d$ 维矢量,即这个波函数是全部格点相干态的直积。两个相干态内积为:
\begin{equation}
	\langle \theta | \phi \rangle = \exp\left(\bar\theta \phi \right).
\end{equation}
对相干态的完备关系为:
\begin{equation}
	\bm 1 = \int e^{-\bar\phi\phi} |\phi\rangle \langle \phi| \prod_i \frac{d\bar\phi_i d\phi_i}{\pi}.
\end{equation}
这组超完备基的积分测度来源于复数高斯积分:
\begin{equation}
	\int d\bar\phi d\phi\ e^{-\bar\phi a \phi} = \frac{\pi}{a}.
\end{equation}
由于相干态满足的关系:
\begin{equation}
	\hat c_i |\phi \rangle = \phi_i |\phi \rangle,
\end{equation}
上述求迹表达式可插入 $M$ 组完备关系:
\begin{equation}
	Z = \int e^{-\sum_m \bar\phi_m\phi_m }
		\left[ \mathcal M(\phi_{1},\phi_{M}) \cdots \mathcal M(\phi_{2},\phi_{1}) \right]
		\prod_{m=1}^{M}\prod_{i_m} \frac{d\bar\phi_{m}^{i_m} d\phi_{m}^{i_m}}{\pi}.
\end{equation}
矩阵元为:
\begin{eqnarray}
	M(\phi_{m+1},\phi_{m}) 
	&\simeq & \langle \phi_{m+1}| 1-\frac{\beta}{M} \hat H(\hat c_i^\dagger, \hat c_j) |\phi_{m} \rangle. \nonumber \\
	&=& \langle \phi_{m+1}| \phi_{m} \rangle -\frac{\beta}{M} H(\bar\phi_{m+1}, \phi_m) \nonumber \\
	&\simeq & 1 + \bar\phi_{m+1}\phi_{m} - \frac{\beta}{M} H(\bar\phi_{m+1}, \phi_m) \nonumber \\
	&\simeq & \exp\left[\bar\phi_{m+1}\phi_{m} - \frac{\beta}{M} H(\bar\phi_{m+1}, \phi_m) \right].
\end{eqnarray}
其中令$M+1=1$,即$\phi_m$满足周期边界条件。现在将积分测度项分配给每一个矩阵$\mathcal M$:
\begin{equation}
	e^{-\bar\phi_{m+1}\phi_{m+1}}\mathcal M(\phi_{m+1},\phi_m)
	\simeq \exp\left[-\bar\phi_{m+1}(\phi_{m+1}-\phi_m) - \frac{\beta}{M} H(\bar\phi_{m+1}, \phi_m) \right].
\end{equation}
在极限 $M \rightarrow \infty$ 下,求和变积分:
\begin{equation}
	\frac{\beta}{M}\sum_m \simeq \int_0^\beta d\tau,
\end{equation}
差分变微分:
\begin{equation}
	\phi_{m+1}-\phi_m \simeq \frac{\beta}{M}\partial_\tau \phi(\tau).
\end{equation}
因此配分函数的表达式为:
\begin{equation}
	Z = \int D[\bar\phi,\phi] \exp\left[-\bar\phi\partial_\tau \phi-H(\bar\phi,\phi) \right].
\end{equation}
其中泛函积分测度应理解成极限:
\begin{equation}
	\int D[\bar\phi,\phi] = \lim_{M\rightarrow \infty} \prod_{m=1}^{M}\prod_{i_m} \frac{d\bar\phi_{m}^{i_m} d\phi_{m}^{i_m}}{\pi}.
\end{equation}
注意极限只对虚时间分割做连续化,而实空间指标可以是连续也可以是离散的,取决于哈密顿量是连续模型还是格点模型。我们这里将哈密顿量视作格点模型,但可以取连续极限化为连续模型。



\subsection*{费米相干态}
\noindent
费米子相干态(未归一化)定义为:
\begin{equation}
	|\psi\rangle \equiv \exp\left(-\sum_i \psi_i \hat c_i^\dagger \right)|0\rangle = \bigotimes_i \left(|0\rangle_i - \psi_i|1\rangle_i \right).
\end{equation}
其中 $\psi_i$ 是 Grassmann 数,相干态之间的内积为:
\begin{equation}
	\langle \psi_1 | \psi_2 \rangle = \exp(\bar\psi_1 \psi_2).
\end{equation}
相干态完备关系为:
\begin{equation}
	\bm 1 = \int e^{-\bar\psi \psi} |\psi\rangle\langle\psi| \prod_i d\bar\psi_i d\psi_i.
\end{equation}
测度来源于 Grassmann 高斯积分:
\begin{equation}
	\int \exp(-\bar\psi a \psi) d\bar\psi d\psi = a.
\end{equation}
同样的方法插入完备关系,最后配分函数的形式与波色情况相同:
\begin{equation}
	Z = \int D[\bar\psi,\psi] \exp\left[-\bar\psi\partial_\tau \psi-H(\bar\psi,\psi) \right].
\end{equation}
不同之处在于泛函积分测度:
\begin{equation}
	\int D[\bar\psi,\psi] = \lim_{M\rightarrow \infty} \prod_{m=1}^{M}\prod_{i_m} d\bar\psi_{m}^{i_m} d\psi_{m}^{i_m}.
\end{equation}
同时注意,费米场泛函积分中所取的场满足反周期边界条件。



\subsection*{自旋相干态}
\noindent
我们首先考虑单个自旋相干态,其定义为:
\begin{equation}
	|g(\theta,\phi) \rangle \equiv e^{-i\phi \hat S_3} e^{-i\theta \hat S_2} \left|\uparrow \right\rangle.
\end{equation}
注意自旋相干态是归一化的。选择 SU(2) 的 Haar 测度 $dg$,完备关系为:
\begin{equation}
	\bm 1 = \int dg\ |g(\theta,\phi)\rangle\langle g(\theta,\phi)|.
\end{equation}
自旋相干态的角参数 $(\theta,\phi)$ 是经典角动量方向,满足关系:
\begin{equation}
	\langle g(\theta,\phi)|\hat{\bm S} |g(\theta,\phi)\rangle 
	= S \bm n(\theta,\phi)
	= S(\sin\theta \cos\phi,\sin\theta \sin\phi,\cos\theta).
\end{equation}
其中 $S$ 是自旋角动量大小,$\bm n$ 是自旋对应经典单位矢量。对单体哈密顿量 $\hat H(\hat S_i)$,仿照前面的方法,得到配分函数为:
\begin{equation}
	Z = \int dg \exp\left[-\langle g|\partial_\tau g\rangle -S\cdot H(\bm n) \right].
\end{equation}
其中,含有虚时间的导数项有特殊的几何意义,记为:
\begin{equation}
	S_{top}=\int_0^\beta \langle g|\partial_\tau g \rangle.
\end{equation}
为此项作用量贡献,首先考虑时间导数:
\begin{eqnarray}
	\partial_\tau |g\rangle = -i \dot{\phi} \hat S_3 |g\rangle
	-i \dot{\theta} e^{-i\phi \hat S_3} \hat S_2 e^{i\phi \hat S_3}|g\rangle.
\end{eqnarray}
利用自旋算符满足的关系:
\begin{equation}
	e^{-i\phi\hat S_i} \hat S_j e^{i\phi\hat S_i}
	= \hat S_j \cos\phi + \epsilon_{ijk}\hat S_k \sin\phi.
\end{equation}
得到:
\begin{equation}
	\int_0^\beta d\tau \langle g|\partial_\tau |g\rangle
	= -i S \int_0^\beta d\tau \dot{\phi} \cos\theta
	= i S \int_0^\beta d\tau \dot{\phi} (1-\cos\theta).
\end{equation}
注意上式中含有 $\dot\theta$ 的项互相抵消,第二个等号利用了周期边界条件。因此,作用量写为:
\begin{equation}
	S[g] = S\int_0^\beta d\tau \left[H(\bm n)+i(1-\cos\theta)\dot{\phi} \right].
\end{equation}
以上的讨论可完全平行地推广到多自旋体系,相应地自旋多体相干态为单体相干态的直积:
\begin{equation}
	|g(\theta,\phi)\rangle = \bigotimes_i |g(\theta_i,\phi_i )\rangle
\end{equation}
此时 $\theta,\phi$ 是 $N^d$ 个分量的矢量。配分函数为:
\begin{equation}
	Z = \int D[g] e^{-S[g]},
\end{equation}
作用量为:
\begin{equation}
	S[g] = S\int_0^\beta d\tau \left[H(\bm n)+i(1-\cos\theta)\partial_\tau \phi \right].
\end{equation}
形式与单体情形保持一致,至少相应的单分量函数 $\bm n,\theta,\phi$ 都变为了多分量函数。




\section*{时空傅立叶变换}
\noindent 
对不显含时间、平移不变的哈密顿量,在动量空间考虑会大大化简问题。这时我们会用到傅立叶变换,然而傅立叶变换前面的归一化系数在不同的地方有着不同的规定。完全不加区分可能在计算过程中算错系数。因此我们在此采用 Atland \& Simons, \textit{Condensed Matter Field Theory} 书中的规定。

对于非相对论的多体系统而言,空间分量和时间分量的地位是不同的。特别的,场的时间分量是真正的连续函数,只是在建立路径积分是为了处理方便将其暂时离散化。而空间分量,本质上是离散的格点,当考虑长波低能行为时,将晶格常数视为无穷小量:$a \rightarrow 0$ ,由此将系统连续化。


\subsection*{函数的傅立叶系数}
\noindent 
首先考虑函数的傅立叶变换,这里通常考虑的是有限体积内定义的连续函数。设函数 $f(\bm r,\tau)$ 是 $d+1$ 维空间的一个函数,其中 $\bm r$ 定义域为 $L^d$ 空间内,$\tau$ 定义域为 $[0,\beta]$,且函数满足周期边界条件(对费米场,满足反周期边界条件)。因此,其动量-频率空间的分量是离散的。傅立叶变换系数规定为:
\begin{eqnarray}
	f_q &=& \int_0^\beta d\tau \int d^d r \ f(\bm r,\tau)e^{-iq\cdot r}, \\
	f(\bm r,\tau) &=& \frac{1}{\beta L^d} \sum_q f_q e^{i q\cdot r}。
\end{eqnarray}
其中4动量 $q\equiv(\bm q,\omega_n)$,内积 $q\cdot r \equiv \bm q\cdot \bm r - \omega_n \tau$,对 $q$ 求和代表对 4 动量求和。

采用这种规定的一个优势是当取热力学极限 $L \rightarrow \infty$ 时,动量积分会变为一个紧凑的形式:
\begin{equation}
	\frac{1}{L^d}\sum_{\bm q} = \int \frac{d^d q}{(2\pi)^d}
\end{equation}
同时,频率求和也往往带上系数 $1/\beta$:
\begin{equation}
	\frac{1}{\beta}\sum_{\omega_n}h(\omega_n)
	= \frac{\xi}{2\pi i} \oint dz\ n_{\xi}(z) h(-iz).
\end{equation}




\subsection*{场的傅立叶系数}
\noindent 
场 $\phi(\bm r,\tau)$ 的傅立叶变换系数定义与函数不同。我们将分别考虑空间和时间的傅立叶变换。对于空间变换,首先重申我们讨论的模型本质是格点模型,场的“空间”坐标本质上是离散的,即:
\begin{equation}
	\phi(\bm r, \tau) = \phi(\bm r_i,\tau),
\end{equation}
其中 $i$ 是格点指标,而场的取值由降算符作用
\begin{equation}
	\hat c_i |\phi(\tau) \rangle = \phi(\bm r_i,\tau) |\phi(\tau) \rangle
\end{equation}
来确定。因此场的傅立叶变换可以由算符的傅立叶变换得到:
\begin{equation}
	\hat c_{\bm p}|\phi(\tau) \rangle = \phi(\bm p,\tau) |\phi(\tau) \rangle.
\end{equation}
而格点升降算符的傅立叶变换的约定为:
\begin{eqnarray}
	\hat c_{\bm p} &=& \frac{1}{N^{d/2}}\sum_i e^{-i\bm p\cdot \bm r_i} \hat c_i, \\
	\hat c_i &=& \frac{1}{N^{d/2}}\sum_{\bm p} e^{i\bm p\cdot \bm r_i} \hat c_{\bm p}.
\end{eqnarray}
这种约定保持了算符的正则对易关系,因此几乎是一种“canonical”的约定。在这种规定下,格点场的空间傅立叶变换规定为:
\begin{eqnarray}
	\phi_{\bm p}(\tau) &=& \frac{1}{N^{d/2}}\sum_i e^{-i\bm p\cdot \bm r_i} \phi(\bm r_i,\tau), \\
	\phi(\bm r_i,\tau) &=& \frac{1}{N^{d/2}}\sum_{\bm p} e^{i\bm p\cdot \bm r_i} \phi_{\bm p}(\tau).
\end{eqnarray}
现在考虑连续模型,这时我们对格点模型做连续近似:
\begin{eqnarray}
	\phi(\bm r,\tau) &\simeq & \frac{1}{a^{d/2}}\phi(\bm r_i,\tau), \\
	a^d \sum_i &\simeq & \int d^d r.
\end{eqnarray}
其中第一个等式中的归一化由
\begin{equation}
	\sum_i \bar\phi(\bm r_i,\tau) \phi(\bm r_i, \tau) = \int d^d r \bar\phi(\bm r,\tau)\phi(\bm r, \tau)
\end{equation}
确定,由此连续场的傅立叶变换:
\begin{eqnarray}
	\phi_{\bm p}(\tau) &=& \frac{1}{L^{d/2}} \int d^d r\ \phi(\bm r,\tau) e^{-i \bm p \cdot \bm r}, \\
	\phi(\bm r,\tau) &=& \frac{1}{L^{d/2}} \sum_{\bm p} \phi_{\bm p}(\tau) e^{i \bm p \cdot \bm r}.
\end{eqnarray}
注意连续极限只改变了场 $\phi(\bm r_i,\tau)$,其傅立叶分量仍然是相同的离散值。

下面考虑频率空间的傅立叶变换。频率空间的傅立叶变换由不同的规定,Atland \& Simons 书中规定为:
\begin{eqnarray}
	\phi_n &=& \frac{1}{\sqrt \beta} \int_0^\beta d\tau\ \phi(\tau) e^{i\omega_n \tau}, \\
	\phi(\tau) &=& \frac{1}{\sqrt \beta} \sum_{n} \phi_n e^{-i\omega_n \tau}.
\end{eqnarray}
在这种规定下,自由粒子作用量
\begin{equation}
	S = \int_0^\beta d\tau \ \bar\phi(\tau)(\partial_\tau + \epsilon) \phi(\tau)
\end{equation}
的傅立叶变换为:
\begin{equation}
	S = \sum_n \bar\phi_n (-i\omega_n + \epsilon) \phi_n.
\end{equation}
注意此时傅立叶分量 $\phi_n$ 带有量纲 $[E^{-1/2}]$,高斯积分时需平衡量纲:
\begin{equation}
	\int D[\bar\phi,\phi]\ e^{-S[\bar\phi,\phi]}
	\simeq \prod_n[\beta(-i\omega_n+\epsilon)]^{-\zeta}.
\end{equation}
以上的等价过程中,我们没有仔细考虑积分测度。由于积分测度可以相差任意大的(无量纲)系数。



\subsection*{算符在动量空间求迹}
\noindent 
对于多个场的路径积分,有时会将部分高斯型的场积掉,在将结果放到指数上,得到有效作用量。考虑某个高斯型场泛函积分:
\begin{equation}
	\int_0^\beta d\tau \int d^d r\ \bar\phi(\bm r,\tau) \hat F \phi(\bm r,\tau),
\end{equation}
其中算符 $\hat F$ 一般含有其他场。我们可以将这个高斯形的场积掉,结果为
\begin{equation}
	\det[\beta \hat F ]^{-\zeta}.
\end{equation}
将行列式放到指数上,相当于在作用量上加上:
\begin{equation}
	\Delta S
	= \ln\det\left[\beta \hat F \right]
	= \Tr \ln \left[\beta \hat F \right].
\end{equation}
在很多时候,我们都可以省略对数中的 $\beta$ 项。因为所需计算的量往往都依赖于对数项的导数,如当我们对算符 $\hat F$ 中包含的某参数 $\alpha$ 求导:
\begin{equation}
	\frac{\partial}{\partial \alpha}\Delta S
	= \Tr \left[\hat F^{-1}\cdot \frac{\partial \hat F}{\partial \alpha}  \right].
\end{equation}
对于求迹的计算,我们采用正交完备基底:
\begin{equation}
	|\bm p, \omega_n \rangle = \frac{1}{\sqrt{\beta L^d}} e^{i\bm p\cdot \bm r - i\omega_n \tau}.
\end{equation}
对任意算符 $\hat F$,其矩阵元为:
\begin{equation}
	\langle \bm p_1, \omega_n| \hat F |\bm p_2, \omega_m \rangle
	= \frac{T}{L^d} \int_0^\beta d\tau \int d^d r\ e^{-i\bm p_1 \cdot \bm r+i\omega_n \tau} \hat F e^{i\bm p_2 \cdot \bm r-i\omega_m \tau}
\end{equation}
比如,对于时空求导算符 $\partial_\tau, \nabla$,其矩阵元为:
\begin{eqnarray}
	\langle \bm p_1, \omega_n| \partial_\tau |\bm p_2, \omega_m \rangle
	&=& -i\omega_n \delta_{mn}\delta{\bm p_1 \bm p_2}, \\
	\langle \bm p_1, \omega_n| \nabla |\bm p_2, \omega_m \rangle
	&=& i\bm p_1 \delta_{mn}\delta{\bm p_1 \bm p_2}.
\end{eqnarray}
而对于函数 $f(\bm r,\tau)$,矩阵元为:
\begin{equation}
	\langle \bm p_1, \omega_n| f(\bm r,\tau) |\bm p_2, \omega_m \rangle
	= \frac{T}{L^d} f_{\bm p_1-\bm p_2, n-m}.
\end{equation}
其中 $f_{\bm p_1-\bm p_2, n-m}$ 是前面定义的普通函数的傅立叶分量。写下算符在动量空间中具体矩阵元,就可以将求迹表达式具体写下来。






\end{document}
